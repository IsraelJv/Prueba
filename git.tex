\documentclass{article}
\usepackage{listings}



\begin{document}
	\title{Comandos b\'asicos de git y github}
	\date{\today}
	\author{Israel Jv}
	\maketitle
	
	\section{Git}
	Para inicializar un repositorio en nuestro ordenador, es necesario abrir una terminal en la carpeta donde se quiera tener el control de versiones e ingresamos lo siguiente:
	
	\begin{lstlisting}[language=Bash]
		git init
	\end{lstlisting}
	
	Nuestros documentos pueden estar en dos estados: sin estar en seguimiento y en espera de commit.
	Para saber el estado de nuestros archivos utilizamos el comando:
	
	\begin{lstlisting}[language=Bash]
		git status
	\end{lstlisting}

	Los commits son puntos de "recuperaci\'on". Supongamos que todo jala a la perfecci\'on y decidimos hacer commit, pero luego metemos algo mal y ya no corre nuestro programa. Es entonces cuando podemos regresarnos al punto en donde todo jala bien y poder intentar de nuevo sin tener problema alguno. \\
	Como dijimos antes, hay dos estados. Para dar seguimiento a todos los archivos utilizaremos el comando:
	
	\begin{lstlisting}[language=Bash]
		git add .
	\end{lstlisting}

	Con esto, al ejecutar el comando de ver el estado con el comando status, veremos que ahora el color de nuestros archivos pasaron de rojo a verde. \\
	Para hacer un punto de recuperaci\'on tenemos que ejecutar el comando: 
	
	\begin{lstlisting}[language=Bash]
		git commit -m "mensaje"
	\end{lstlisting}
	
	El mensaje que pongamos tiene que ser corto y significativo para que tanto t\'u como los dem\'as miembros del equipo sepan cuales fueron los cambios que realizaste. \\
	Estos 
	
	
\end{document}